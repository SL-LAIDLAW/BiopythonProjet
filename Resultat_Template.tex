\documentclass[a4paper]{article}
\usepackage{moreverb,url}
\usepackage{float}
\usepackage{verbatim}
\usepackage[francais]{babel}
\usepackage[utf8]{inputenc}
\usepackage[T1]{fontenc}


\RequirePackage{graphicx}
\RequirePackage{latexsym,ifthen,rotating,calc,textcase,booktabs,color,endnotes}
\RequirePackage{amsfonts,amssymb,amsbsy,amsmath,amsthm}
\RequirePackage[errorshow]{tracefnt}

\usepackage{multicol}

\usepackage{times}
\usepackage[scaled=.95]{helvet}

\usepackage[colorlinks,bookmarksopen,bookmarksnumbered,citecolor=red,urlcolor=red]{hyperref}



\usepackage[text={174mm,258mm},%
papersize={210mm,297mm},%
columnsep=12pt,%
headsep=21pt,%
centering]{geometry}
\usepackage{ftnright}




\begin{document}

\noindent{\huge{\textbf{Résultats du Projet HLIN404 pour gène [geneInteret*] chez [especegeneInteret*]}}}\\

\noindent{\Large{\textbf{[NomUtilisateur*]}}}
\vspace*{7mm}


\begin{multicols}{2}[
\noindent\textbf{Objectif : }Ce document a comme raison d'être, l'affichage des résultats BLAST d'un gène d'intérêt — ici il s'agit du gène [geneInteret*] — ainsi que des informations complémentaires, comme le BLAST des top quatre résultats trouvé dernièrement, ainsi que la mise en place d'un arbre phylogénétique.
\vspace*{1mm}]

\section{Introduction}
Tout d'abord nous avons choisi avec le script bash "Bioinfo.sh", le gène [geneInteret*] à analyser. Ce logiciel a comme but de trouver quels modèles animaux seront les plus adapté pour étudier un gène donnée. Nous avons donc construit le logiciel d'un tel sorte à ce que :
\begin{enumerate}
\item[(i)] On BLAST notre séquence du gene d'intérêt sur le NCBI, et on récupère les 4 séquences qui ont les scores HSP le plus élevés.

\item[(ii)] Ensuite on crée un base de données à partir de la séquence du gène [geneInteret*] du [especegeneInteret*] ainsi que les séquences des 4 espèces les plus proches.

\item[(iii)] On BLAST chacun de ces 4 séquences, en local en utilisant ce base de données, et on trie les résultats afin de ne pas obtenir de duplicatas.

\item[(iv)] À partir du multifasta crée lors de la recherche des quatre espèces les plus proche à [especegeneInteret*], on construit un arbre phylogénétique.
\end{enumerate}

\section{BLAST initiale sur NCBI}
Avec le fasta de notre gène d'intérêt en main, on l'as envoyé au NCBI pour être BLASTé et on a récupéré les résultats suivants :
\begin{table}[H]
\small\sf\centering
\caption{Résultats du BLAST NCBI sur gène d'intérêt de l'espèce [especegeneInteret*]}
\begin{tabular}{lll}
\toprule
Bit-Score&Espèce&Identifiant\\
\midrule
[TableaugeneInteret*]
\bottomrule
\end{tabular}\\[10pt]
\end{table}

\section{BLAST local des autres espèces}
À partir des réponses trouvées dans la dernière partie, on a analysé avec un BLAST en local, chacun des quatre espèces trouvés

\subsection{[especegene1*]}
\begin{table}[H]
\small\sf\centering
\caption{Résultats du BLAST local pour le gène d'intérêt de l'espèce [especegene1*]}
\begin{tabular}{lll}
\toprule
Bit-Score&Espèce&Identifiant\\
\midrule
[Tableaugene1*]
\bottomrule
\end{tabular}\\[10pt]
\end{table}

\subsection{[especegene2*]}
\begin{table}[H]
\small\sf\centering
\caption{Résultats du BLAST local pour le gène d'intérêt de l'espèce [especegene2*]}
\begin{tabular}{lll}
\toprule
Bit-Score&Espèce&Identifiant\\
\midrule
[Tableaugene2*]
\bottomrule
\end{tabular}\\[10pt]
\end{table}

\subsection{[especegene3*]}
\begin{table}[H]
\small\sf\centering
\caption{Résultats du BLAST local pour le gène d'intérêt de l'espèce [especegene3*]}
\begin{tabular}{lll}
\toprule
Bit-Score&Espèce&Identifiant\\
\midrule
[Tableaugene3*]
\bottomrule
\end{tabular}\\[10pt]
\end{table}

\subsection{[especegene4*]}
\begin{table}[H]
\small\sf\centering
\caption{Résultats du BLAST local pour le gène d'intérêt de l'espèce [especegene4*]}
\begin{tabular}{lll}
\toprule
Bit-Score&Espèce&Identifiant\\
\midrule
[Tableaugene4*]
\bottomrule
\end{tabular}\\[10pt]
\end{table}

\end{multicols}


\section{Arbre Phylogénétique}
\begin{figure}[H]
\begin{verbatim}
[monarbrephylo*]
\end{verbatim}
\caption{Arbre Phylogénétique}
\label{arbrephylogénétique}
\end{figure}



\end{document}
