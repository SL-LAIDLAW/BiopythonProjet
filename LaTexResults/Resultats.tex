\documentclass[a4paper]{article}
\usepackage{moreverb,url}
\usepackage{float}
\usepackage{verbatim}
\usepackage[francais]{babel}
\usepackage[utf8]{inputenc}
\usepackage[T1]{fontenc}


\RequirePackage{graphicx}
\RequirePackage{latexsym,ifthen,rotating,calc,textcase,booktabs,color,endnotes}
\RequirePackage{amsfonts,amssymb,amsbsy,amsmath,amsthm}
\RequirePackage[errorshow]{tracefnt}

\usepackage{multicol}

\usepackage{times}
\usepackage[scaled=.95]{helvet}

\usepackage[colorlinks,bookmarksopen,bookmarksnumbered,citecolor=red,urlcolor=red]{hyperref}



\usepackage[text={174mm,258mm},%
papersize={210mm,297mm},%
columnsep=12pt,%
headsep=21pt,%
centering]{geometry}
\usepackage{ftnright}




\begin{document}

\noindent{\huge{\textbf{Résultats du Projet HLIN404 pour gène DEC2 chez Homo sapiens}}}\\

\noindent{\Large{\textbf{Sean Louis LAIDLAW}}}
\vspace*{7mm}


\begin{multicols}{2}[
\noindent\textbf{Objectif : }Ce document a comme raison d'être, l'affichage des résultats BLAST d'un gène d'intérêt — ici il s'agit du gène DEC2 — ainsi que des informations complémentaires, comme le BLAST des top quatre résultats trouvé dernièrement, ainsi que la mise en place d'un arbre phylogénétique.
\vspace*{1mm}]

\section{Introduction}
Tout d'abord nous avons choisi avec le script bash "Bioinfo.sh", le gène DEC2 à analyser. Ce logiciel a comme but de trouver quels modèles animaux seront les plus adapté pour étudier un gène donnée. Nous avons donc construit le logiciel d'un tel sorte à ce que :
\begin{enumerate}
\item[(i)] On BLAST notre séquence du gene d'intérêt sur le NCBI, et on récupère les 4 séquences qui ont les scores HSP le plus élevés.

\item[(ii)] Ensuite on crée un base de données à partir de la séquence du gène DEC2 du Homo sapiens ainsi que les séquences des 4 espèces les plus proches.

\item[(iii)] On BLAST chacun de ces 4 séquences, en local en utilisant ce base de données, et on trie les résultats afin de ne pas obtenir de duplicatas.

\item[(iv)] À partir du multifasta crée lors de la recherche des quatre espèces les plus proche à Homo sapiens, on construit un arbre phylogénétique.
\end{enumerate}

\section{BLAST initiale sur NCBI}
Avec le fasta de notre gène d'intérêt en main, on l'as envoyé au NCBI pour être BLASTé et on a récupéré les résultats suivants :
\begin{table}[H]
\small\sf\centering
\caption{Résultats du BLAST NCBI sur gène d'intérêt de l'espèce Homo sapiens}
\begin{tabular}{lll}
\toprule
Bit-Score&Espèce&Identifiant\\
\midrule
5748.62 & Macaca fascicularis & 982280190\\
5732.39 & Macaca nemestrina & 795411137\\
5654.85 & Chlorocebus sabaeus & 635064225\\
5059.74 & Cebus capucinus & 1044422013\\
4650.37 & Microcebus murinus & 1149877709\\

\bottomrule
\end{tabular}\\[10pt]
\end{table}

\section{BLAST local des autres espèces}
À partir des réponses trouvées dans la dernière partie, on a analysé avec un BLAST en local, chacun des quatre espèces trouvés

\subsection{Macaca fascicularis}
\begin{table}[H]
\small\sf\centering
\caption{Résultats du BLAST local pour le gène d'intérêt de l'espèce Macaca fascicularis}
\begin{tabular}{lll}
\toprule
Bit-Score&Espèce&Identifiant\\
\midrule
8935.21 & Macaca nemestrina & 795411137\\
6669.37 & Chlorocebus sabaeus & 635064225\\
5373.02 & Cebus capucinus & 1044422013\\
490.482 & Homo sapiens & 221044669\\

\bottomrule
\end{tabular}\\[10pt]
\end{table}

\subsection{Macaca nemestrina}
\begin{table}[H]
\small\sf\centering
\caption{Résultats du BLAST local pour le gène d'intérêt de l'espèce Macaca nemestrina}
\begin{tabular}{lll}
\toprule
Bit-Score&Espèce&Identifiant\\
\midrule
8935.21 & Macaca fascicularis & 982280190\\
6643.52 & Chlorocebus sabaeus & 635064225\\
5350.86 & Cebus capucinus & 1044422013\\
484.943 & Homo sapiens & 221044669\\

\bottomrule
\end{tabular}\\[10pt]
\end{table}

\subsection{Chlorocebus sabaeus}
\begin{table}[H]
\small\sf\centering
\caption{Résultats du BLAST local pour le gène d'intérêt de l'espèce Chlorocebus sabaeus}
\begin{tabular}{lll}
\toprule
Bit-Score&Espèce&Identifiant\\
\midrule
6669.37 & Macaca fascicularis & 982280190\\
6643.52 & Macaca nemestrina & 795411137\\
5302.85 & Cebus capucinus & 1044422013\\
490.482 & Homo sapiens & 221044669\\

\bottomrule
\end{tabular}\\[10pt]
\end{table}

\subsection{Cebus capucinus}
\begin{table}[H]
\small\sf\centering
\caption{Résultats du BLAST local pour le gène d'intérêt de l'espèce Cebus capucinus}
\begin{tabular}{lll}
\toprule
Bit-Score&Espèce&Identifiant\\
\midrule
5373.02 & Macaca fascicularis & 982280190\\
5350.86 & Macaca nemestrina & 795411137\\
5302.85 & Chlorocebus sabaeus & 635064225\\
479.403 & Homo sapiens & 221044669\\

\bottomrule
\end{tabular}\\[10pt]
\end{table}

\end{multicols}


\section{Arbre Phylogénétique}
\begin{figure}[H]
\begin{verbatim}
  __________________________________________________________________ Homo_sapiens
_|
______________________________________________________________________ Chlorocebus_sabaeus
 |        |
          |          ______________________________________________ Cebus_capucinus
____________________|_________________________________________________
                    |                                             , Macaca_nemestrina
                    |_____________________________________________|
                                                                  | Macaca_fascicularis


\end{verbatim}
\caption{Arbre Phylogénétique}
\label{arbrephylogénétique}
\end{figure}



\end{document}
